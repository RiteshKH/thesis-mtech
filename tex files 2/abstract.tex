

\begin{center}
	\huge{\textbf{Abstract}}
\end{center}

Unmanned aerial vehicles have gathered attention recently in various domains owing to their robustness and efficiency, compared to conventional aircrafts. Quadrotors specially are very effective for surveillance purposes. This however comes with a bundle of issues, such as determining the position of the subject, high computation power for detection, obstacle avoidance, control related issues in case of occlusion, etc. This thesis proposes a simple approach to detect and track a desired object or person, and use an UAV to follow it autonomously. The approach seeks zero dependency on the moving target to transmit it’s position in any form. The method chosen for target recognition is vision-based using a common camera mounted on the quadrotor and deep learning techniques were employed for detection. A simple follow control algorithm is then used to enable the vehicle to follow any path which the target acquires autonomously. Erratic behaviour of human movement was considered and smooth control inputs were generated by employing Kalman Filter/Tracking techniques. This also helped it to function appreciably in occluded and crowded environments. The use of a remote server for the detections task further speeds up the process, since quadrotors have low payload carrying capacity.

To evaluate the performance of the proposed approach, simulation studies are performed for 3D cases. These simulations involves a human model, moving in a fixed trajectory for now, and quadrotor model performing the track and follow operation. Different environment scenarios, involving different trajectories for the human, are also tested upon.




















